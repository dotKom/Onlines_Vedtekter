\setcounter{chapter}{-1}
\chapter{Definisjoner}
\vspace{23pt}

\begin{itemize}
    \item Simpelt flertall: Det er flere stemmer for enn mot. Blanke stemmer teller ikke. 
    \item Kvalifisert 2/3 flertall: Mer enn 2/3 av de stemmeberettigede tilstede som avgir stemme stemmer for. Blanke stemmer teller ikke. 
    \item Kvalifisert 3/4 flertall: Mer enn 3/4 av de stemmeberettigede tilstede som avgir stemme stemmer for. Blanke stemmer teller ikke.
    \item Akklamasjon: Et felles klapp fra forsamlingen som signaliserer bred enighet. Akklamasjon kan tas i bruk av ordstyrer dersom de anser at det er enighet i salen. Det skal gis rom for å uttrykke sin misnøye innen rimelig tid og enhver stemmeberettiget person tilstede kan kreve at det gjennomføres en fullstendig avstemning.
    \item Stemmetall: Det antallet, er den andelen av, stemmer som en person må oppå for å bli valgt. 
  \end{itemize}

%\vspace{40pt}