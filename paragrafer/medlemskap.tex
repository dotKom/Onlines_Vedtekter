\chapter{Medlemskap}\label{chapter:medlemskap}

\section{Medlemmer}{\label{sec:medlemmer}
\vspace{23pt}

Studenter ved følgende studier ved NTNU har rett til å bli medlem av linjeforeningen:
\begin{liste}
	\item Bachelor i informatikk (BIT)
	\item Master i informatikk (MIT)
	\item Master of Science in Informatics (MSIT)
    \item PhD-kandidater ved Institutt for datateknologi og informatikk ved NTNU, som har tatt en bachelor eller master i informatikk
\end{liste}
}
\section{Medlemmers rettigheter}
\vspace{23pt}
Alle medlemmer av linjeforeningen har møte-, tale- og stemmerett på linjeforeningens generalforsamling. Alle medlemmer i linjeforeningen har rett til å levere saks- og vedtektsforslag til generalforsamlingen. Alle medlemmer av linjeforeningen har rett til å søke Hovedstyret om å starte opp interessegrupper. I tillegg har alle medlemmer rett til å delta på de arrangementer som arrangeres for medlemmene, gitt at arrangementets spesifikasjoner passer det aktuelle medlemmet. Medlemmer av linjeforeningen kan kreve innsyn i linjeforeningens regnskap og kan få innsyn i ikke-konfidensielle vedtak.

\subsection{Livstidsmedlemmer}{
Et medlem på livstid kan delta på sosiale arrangementer arrangert for alle medlemmer. I tillegg har medlemmer på livstid møte- og talerett på generalforsamlingen, men ikke stemmerett. Livstidsmedlemskapsrettigheter tiltrår ikke før etter man har avsluttet medlemskap jf. §\ref{sec:medlemmer}.

Følgende er medlemmer av linjeforeningen på livstid:
\begin{liste}
	\item Ridder av det Indre lager
\end{liste}

}
%----
\section{Utmelding, opphør, ekskludering}
\vspace{23pt}
Alle medlemmer kan melde seg ut av linjeforeningen ved å melde fra skriftlig til Hovedstyret. Utmelding gjelder fra utmeldingsdato. Når en person ikke lenger oppfyller kravene for medlemskap i linjeforeningen vil medlemskapet opphøre.

Hovedstyret kan, ved skjellig grunn, utestenge medlemmer fra linjeforeningen, midlertidig eller permanent, dersom det er kvalifisert flertall for det.


%----
\section{Adgang for andre til å bli medlem}
\vspace{23pt}
På spesielt grunnlag kan Hovedstyret, ved kvalifisert flertall, tillate andre å bli medlem. Dette gjelder primært studenter ved andre fakulteter og linjer som har informatikk som et av sine hovedstudier.
