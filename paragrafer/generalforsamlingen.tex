\chapter{Generalforsamlingen}

Generalforsamlingen er linjeforeningens øverste organ og er uavhengig av gjeldende hovedstyrevedtak. Generalforsamlingen avholdes årlig i løpet av vårsemesteret.\newline

Den ordinære generalforsamlingen skal behandle årsmelding, innsendte saker, vedtektsendringer, valg og regnskap for foregående år, valg av nytt hovedstyre og valg av ny valgkomité.

Hovedstyret kan i etterkant av Generalforsamlingen gjøre redaksjonelle endringer i vedtektene.  

%----
\section{Frister}
\label{sec:frister}
\begin{liste}
	\item Innkalling skal sendes ut til medlemmene senest \emph{fire (4) uker}  før \mbox{generalforsamlingen} skal avholdes.
	\item Saksforslag og forslag til vedtektsendringer sendes Hovedstyret senest \emph{to (2) uker} før generalforsamlingen skal avholdes.
	\item Fullstendig saksliste med vedtektsendringer skal tilgjengeliggjøres senest en (1) uke før møtedato. Denne skal også inneholde årsmelding, revidert regnskap, vedtatt budsjett for året og eventuelle andre relevante sakspapirer.
	\item Referat fra generalforsamlingen skal underskrives av paraferer og sendes \linebreak medlemmene eller gjøres tilgjengelig for medlemmene senest 14 dager etter generalforsamlingen.
\end{liste}


%----
\section{Ekstraordinær generalforsamling}
\vspace{23pt}
Det skal kalles inn til ekstraordinær generalforsamling dersom Hovedstyret eller det minste av 1/8 av medlemmene og ti (10) medlemmer ønsker det. Fristene for å kalle inn til ekstraordinær generalforsamling er halvert i forhold til fristene for ordinær generalforsamling, jf. \ref{sec:frister}\newline

Ekstraordinær generalforsamling skal kun behandle den (de) saken(e) som står på dagsorden for den ekstraordinære generalforsamlingen
 

%----
\section{Organisering} \label{sec:organisering}
\vspace{23pt}
Ved generalforsamling er disse vervene nødvendig: 

\begin{liste}
	\item Ordstyrer
	\item To referenter - skriver referat under generalforsamling og samarbeider om \mbox{renskriving}
	\item Minst to til tellekorps - teller opp stemmer ved avstemming
	\item To paraferer - godkjenner referat fra generalforsamling og de endrede vedtektene i etterkant av generalforsamlingen
	\item Tre valgkomitémedlemmer - har et ansvar for å foreslå kandidater til neste års generalforsamling
	
\end{liste}

%----
\newpage
\section{Beslutningsdyktighet og avstemming}
\vspace{23pt}

For at en generalforsamling skal være beslutningsdyktig må det laveste mellom 15 medlemmer og 1/5 av medlemmene ha møtt opp.\newline
Verken forhåndsstemming eller fullmakter er tillatt å bruke ved avstemning.

\subsection{Saker}

Alle saker på generalforsamlingen avgjøres ved alminnelig flertall, med unntak av vedtektsendringer. 

\subsection{Vedtekter}

Vedtektsendringer avgjøres med 2/3 kvalifisert flertall. 




%----
\section{Stemmeberettigelse og talerett}
\vspace{23pt}
Ethvert medlem av linjeforeningen har talerett ved generalforsamlingen.
Ethvert medlem av linjeforeningen som er tilstede når generalforsamlingen godkjenner stemmeberettigede har rett til å stemme.\newline

Medlemmer av linjeforeningen som ikke har mulighet til å møte i tide plikter å informere Hovedstyret og oppgi en tilstrekkelig grunn til forsinkelse. Generalforsamlingen kan vedta å gi disse personene stemmerett samtidig som stemmeberettigede godkjennes.

Generalforsamlingen kan vedta å gi medlemmer av linjeforeningen som kommer for sent, og ikke har informert om dette, stemmerett når vedkommende ankommer.


%----
\section{Gjennomføring av valg}{
\vspace{23pt}
Dersom det er mer enn en kandidat til et verv skal det avholdes hemmelig valg for vervet. Man kan stemme på “ingen” for å vise at man ikke ønsker noen av kandidatenene. Stemmetallet for personvalg der det skal fylles én stilling er 50\% av avgitte stemmer, blanke stemmer teller ikke som avgitte stemmer.\newline

Dersom ingen av kandidatene oppnår stemmetallet fjernes den kandidaten med færrest stemmer og en ny valgrunde gjennomføres. Dersom ingen av kandidatene oppnår stemmetallet, og det er stemmelikhet på de kandidatene som har færrest stemmer, skal det gjennomføres en ny valgrunde, med samme kandidater.\newline

Innehavere av verv sitter inntil endt generalforsamling hvor det er gjennomført et godkjent valg for det respektive vervet. Dersom generalforsamlingen ikke klarer å gjennomføre et valg må det kalles inn til ekstraordinær forsamling innen tre dager etter endt ordinær generalforsamling.\newline

Ved opptelling av hemmelig valg skal tellekorps sitte i salen.


	\subsection{Fraskrivelse av rett til å stille til valg} {
	Personer som er innstilt med et av følgende verv under Generalforsamlingen fraskriver seg sin rett til å stille til alle andre valg.
	\begin{liste}
		\item Ordstyrer
		\item Tellekorps
	\end{liste}
	Med å stille til andre valg menes det at man ikke kan stille, eller bli nominert, til andre verv under Generalforsamlingen og valg til Hovedstyret \newline

    Medlemmer som ble valgt til valgkomiteen ved forrige generalforsamling, kan ikke stille til verv i Hovedstyret.

	}
}
